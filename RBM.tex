\documentclass[a4paper]{article}

\linespread{1.1}
\usepackage[utf8]{inputenc} 
\usepackage[T1]{fontenc}
\usepackage[francais]{babel}
\usepackage{amsmath}
\usepackage{amsfonts}
\usepackage{amssymb}
\usepackage{graphicx}
\usepackage{lmodern}
\usepackage{microtype}
\usepackage{hyperref}
\usepackage[margin=1.8cm]{geometry}
\usepackage{pgf,tikz}
\usepackage{mathrsfs}
\usetikzlibrary{arrows}


% Structure

\newcounter{c}
\newcounter{d}
\newcounter{r}
\newcounter{e}

\newcommand{\defi}{\subparagraph{Definition \arabic{c}.\arabic{d} :}\stepcounter{d}}
\newcommand{\prop}{\subparagraph{Proposition \arabic{c}.\arabic{r} :}\stepcounter{r}}
\newcommand{\thm}{\subparagraph{Theorem \arabic{c}.\arabic{r} :}\stepcounter{r}}
\newcommand{\demo}{\subparagraph{Proof}}
\newcommand{\cor}{\subparagraph{Corollary \arabic{c}.\arabic{r} :}\stepcounter{r}}
\newcommand{\lem}{\subparagraph{Lemma \arabic{c}.\arabic{r} :}\stepcounter{r}}
\newcommand{\rem}{\subparagraph{Remark :}}
\newcommand{\chapitre}[1]{\stepcounter{c}\setcounter{e}{0}\setcounter{d}{0}\setcounter{r}{0}\noindent\textbf{\Large#1}\\}
\newcommand{\eq}[1]{\stepcounter{e}\begin{equation}#1\tag{\arabic{c}.\arabic{e}}\end{equation}}

% Notations

\newcommand{\Q}{\mathbb{Q}}
\newcommand{\Z}{\mathbb{Z}}
\newcommand{\N}{\mathbb{N}}
\newcommand{\R}{\mathbb{R}}
\newcommand{\C}{\mathbb{C}}
\newcommand{\E}[1]{\mathbb E\left(#1\right)}
\newcommand{\sph}{\mathbb{S}}
\newcommand{\p}{{\cal{P}}}
\newcommand{\fsp}{{\cal{F}}}
\newcommand*{\qed}{\hfill\ensuremath{\square}}
\newcommand{\x}{\mathbf x}
\newcommand{\y}{\mathbf y}
\newcommand{\e}{\mathbf e}
\newcommand{\scal}[2]{\langle#1,#2\rangle}
\newcommand{\trans}{^\text{T}\!}
\newcommand{\dt}{\frac d{dt}}

\newcommand{\dpart}[2]{\displaystyle\frac{\partial#1}{\partial #2}}

\newcommand{\nor}[2]{{\cal N}(#1,#2)}
\newcommand{\mat}[2]{{\cal M}_{#1\times#2}(\R)}

\newcommand{\X}{\mathbf X}

\renewcommand{\leq}{\leqslant}
\renewcommand{\geq}{\geqslant}

\newcommand{\vertiii}[1]{{\left\vert\kern-0.25ex\left\vert\kern-0.25ex\left\vert #1\right\vert\kern-0.25ex\right\vert\kern-0.25ex\right\vert}}


\usepackage{listings}
\lstset{
	language=MatLab,
	basicstyle=\color{black}\ttfamily,
	breakatwhitespace=false,
	breaklines=true,
	keywordstyle=\color{blue}\ttfamily,
	stringstyle=\color{red}\ttfamily,
	commentstyle=\color{green}\ttfamily,
	stepnumber=1,
         numbers=left,
	numberstyle=\tiny\color{black}
}

\begin{document}

\begin{center}\Large\bf RBM\end{center}

\paragraph{FEM background}~


Let $X$ be a function Hilbert space such that $H_0^1(\Omega)\subset X\subset H^1(\Omega)$, with $\Omega$ a domain in $\R^d$, d=1,2,3, endowed with an inner product $(\cdot,\cdot)_X$ and the subsequent norm $\|\cdot\|_X$. Given a parametric bilinear form $a(\cdot,\cdot;\cdot):X\times X\times\mathcal D\to\R$ and a parametric linear form $f:X\times\mathcal D\to\R$, we want to find $u\in X$ such that for all $v\in X$ and a fixed $\mu\in\mathcal D$ :

$$a(u,v;\mu)=f(v;\mu)$$

\noindent In many cases, we want to compute an output quantity $s(\mu)=l(u)$ where $u\in X$ is the solution of the previous problem for the prescribed $\mu$ and $l$ is a linear functional. Whenever $l(u)=f(u;\mu)$, we say the problem is \emph{compliant}.


The point of FEM is to discretize $\Omega$ and define a FE solution as its nodal values. It follows that the solution we are looking for lives in an $\mathcal N$ dimensional space $X^\mathcal N$. Let $\left(\phi_i\right)_{i=1,...,\mathcal N}$ be a basis of the FEM space $X^\mathcal N$. We build the matrices :

$$A^\mathcal N(\mu)=(a(\phi_i,\phi_j;\mu))_{i,j=1,...,\mathcal N}$$
$$F^\mathcal N(\mu)=(f(\phi_i;\mu))_{i=1,...,\mathcal N}$$
$$L^\mathcal N(\mu)=(l(\phi_i;\mu))_{i=1,...,\mathcal N}$$

\noindent It follows we just have to solve the linear system :

$$\left(A^\mathcal N(\mu)\right)^T\underline{u}^\mathcal N=F^\mathcal N(\mu)$$

\noindent where obviously $\underline u^\mathcal N\in\R^\mathcal N$ is the vector representing $u^\mathcal N\subset X^\mathcal N$ in the basis $\left(\phi_i\right)_{i=1,...,\mathcal N}$.


This method can be quite expensive if we wish to compute $\underline u^\mathcal N$ for several values of $\mu$. The idea is to reduce our model's order.

\paragraph{RBM}~

Let $S_N\subset\mathcal D$ be a set of $N<<\mathcal N$ parameters. We can choose $S_N$ built thanks to a greedy algorithm fed with a fine sampling of $\mathcal D$ and a suitable approximation error upper bound functional $\Delta$. This is not the only possibility.


We compute $W_N=\{\underline u^\mathcal N\}_{i=1,...,N}$ and apply the Gram-Schmidt orthonormalization procedure resulting in a basis $\left(\xi_i\right)_{i=1,...,N}$ for our RB space $X^N=\text{span}\left(W_N\right)$. We now build the matrix $Z_N$ by packing the basis vectors $\xi_i$ together and set :

$$A^N(\mu)=Z_N^TA^\mathcal N(\mu)Z_N$$
$$F^N(\mu)=Z_N^TF^\mathcal N(\mu)$$
$$L^N(\mu)=Z_N^TL^\mathcal N(\mu)$$

This yields the RB problem :

$$\left(A^N(\mu)\right)^T\underline u^N=F^N(\mu)$$

\noindent where $\underline u^N\in\R^N$ is the vector representing our solution $u^N$ in the basis $W_N$. We hope $u^N$ is a sufficient approximation of $u$ for a parameter domain of interest within $\mathcal D$.


Suppose $a(\cdot,\cdot;\mu)$ is positive definite and let $\vertiii\cdot_\mu$ be the energy norm defined by :

$$\vertiii u_\mu^2=a(u,u;\mu)$$

\newpage
\noindent We have the following optimality results :

$$\vertiii{u-u^N}_\mu=\inf_{w^N\in X^N}\vertiii{u-w^N}_\mu$$
$$\|u-u^N\|_X\leq\sqrt{\frac{\gamma(\mu)}{\alpha(\mu)}}\inf_{w^N\in X^N}\|u-w^N\|_X$$

\bigskip

When the bilinear form $a$ admits an affine decomposition, we can quicken the computations with a suitable offline/online strategy. Suppose we can write :

$$a(u,v;\mu)=\sum_{q=1}^{Q_a}\theta^q_a(\mu)a^q(u,v)$$
$$f(v;\mu)=\sum_{q=1}^{Q_f}\theta^q_f(\mu)f^q(v)$$

The point is to compute the matrices $A^q=\left(a(\xi_i,\xi_j)\right)_{i,j=1,...,N}$ and $F^q=\left(f^q(\xi_i)\right)_{i=1,...,N}$ offline and to store them. Solving the system now reduces to the calculation of the $\theta$ coefficients. The corresponding FEM solution can be rebuilt easily : $Z_N\underline u^N$.


Moreover, if $a(\cdot,\cdot;\mu)$ is continuous and coercive for all $\mu$ with corresponding coefficients $\gamma(\mu)$ and $\alpha(\mu)$, we have an upper bound for the condition number of $A^N(\mu)$ (related to the error propagation) :

$$\text{cond}\left(A^N(\mu)\right)\leq\frac{\gamma(\mu)}{\alpha(\mu)}$$

It can be proven and experimentally shown that the online step does not depend on $\mathcal N$.


\end{document}