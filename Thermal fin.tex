% Configuration

\documentclass[a4paper]{article}

\linespread{1.1}

\usepackage[utf8]{inputenc} 
\usepackage[T1]{fontenc}
\usepackage[francais]{babel}
\usepackage{amsmath}
\usepackage{amsfonts}
\usepackage{graphicx}
\usepackage{lmodern}
\usepackage{microtype}
\usepackage{hyperref}
\usepackage[margin=1.8cm]{geometry}
\usepackage{pgf,tikz}
\usepackage{mathrsfs}
\usetikzlibrary{arrows}


% Structure

\newcounter{c}
\newcounter{d}
\newcounter{r}
\newcounter{e}

\newcommand{\defi}{\subparagraph{D\'efinition \arabic{c}.\arabic{d} :}\stepcounter{d}}
\newcommand{\prop}{\subparagraph{Proposition \arabic{c}.\arabic{r} :}\stepcounter{r}}
\newcommand{\thm}{\subparagraph{Th\'eor\`eme \arabic{c}.\arabic{r} :}\stepcounter{r}}
\newcommand{\demo}{\subparagraph{D\'emonstration}}
\newcommand{\cor}{\subparagraph{Corollaire \arabic{c}.\arabic{r} :}\stepcounter{r}}
\newcommand{\lem}{\subparagraph{Lemme \arabic{c}.\arabic{r} :}\stepcounter{r}}
\newcommand{\rem}{\subparagraph{Remarque :}}
\newcommand{\chapitre}[1]{\stepcounter{c}\setcounter{e}{0}\setcounter{d}{0}\setcounter{r}{0}\noindent\textbf{\Large#1}\\}
\newcommand{\eq}[1]{\stepcounter{e}\begin{equation}#1\tag{\arabic{c}.\arabic{e}}\end{equation}}

% Notations

\newcommand{\Q}{\mathbb{Q}}
\newcommand{\Z}{\mathbb{Z}}
\newcommand{\N}{\mathbb{N}}
\newcommand{\R}{\mathbb{R}}
\newcommand{\C}{\mathbb{C}}
\newcommand{\E}[1]{\mathbb E\left(#1\right)}
\newcommand{\sph}{\mathbb{S}}
\newcommand{\p}{{\cal{P}}}
\newcommand{\fsp}{{\cal{F}}}
\newcommand*{\qed}{\hfill\ensuremath{\square}}
\newcommand{\x}{\mathbf x}
\newcommand{\y}{\mathbf y}
\newcommand{\e}{\mathbf e}
\newcommand{\scal}[2]{\langle#1,#2\rangle}
\newcommand{\trans}{^\text{T}\!}

\newcommand{\nor}[2]{{\cal N}(#1,#2)}
\newcommand{\mat}[2]{{\cal M}_{#1\times#2}(\R)}

\newcommand{\bu}{\mathbf u}
\newcommand{\bv}{\mathbf v}

\date{}
\author{Philippe Ricka}
\title{}

\newcommand{\saut}{\vspace{0.5em}}

\begin{document}

%\maketitle

\begin{center}\huge Thermal fin\end{center}

\bigskip

\chapitre{Finite Element Approximation}


\paragraph{(a)}We consider the five PDEs :

$$-k^i\Delta\bu^i=0,i=0,...,4$$


\noindent we wish to write the variational formulation. We apply rightly the inner product by any standard test function $\bv\in X^e$ then integrate on $\Omega^i$ for each $i=0,...,4$ before taking the sum of these five terms :

$$-\sum_{i=0}^4k^i\int_{\Omega^i}\Delta\bu^i\cdot\bv=0$$


Considering $k^0=1$ and integrating by parts, we get :

$$\sum_{i=0}^4k^i\left[\int_{\Omega^i}\nabla\bu^i\cdot\nabla\bv-\int_{\Gamma^i}(\nabla\bu^i\cdot\mathbf n^i)\cdot\bv\right]=0$$


But since $\Gamma^0=\Gamma_{root}\cup\Gamma^0_{int}\cup\Gamma^0_{ext}$ and $\Gamma^i=\Gamma^i_{int}\cup\Gamma^i_{ext}$ for $i=0,...,4$, we may write :

$$\begin{array}{rl}0=&\displaystyle\sum_{i=0}^4k^i\int_{\Omega^i}\nabla\bu^i\cdot\nabla\bv-\sum_{i=1}^4k^i\left[\int_{\Gamma^i_{int}}(\nabla\bu^i\cdot\mathbf n^i)\cdot\bv+\int_{\Gamma_{ext}^i}(\nabla\bu^i\cdot\mathbf n^i)\cdot\bv\right]\\
&\\
&\displaystyle-k^0\left[\int_{\Gamma_{root}}(\nabla\bu^0\cdot\mathbf n^0)\cdot\bv+\int_{\Gamma^0_{ext}}(\nabla\bu^0\cdot\mathbf n^0)\cdot\bv\right]-k^0\int_{\Gamma^0_{int}}(\nabla\bu^0\cdot\mathbf n^0)\cdot\bv\\
&\\
=&\displaystyle\sum_{i=0}^4k^i\int_{\Omega^i}\nabla\bu^i\cdot\nabla\bv-\left[\sum_{i=1}^4k^i\int_{\Gamma^i_{int}}(\nabla\bu^i\cdot\mathbf n^i)\cdot\bv+\int_{\Gamma^0_{int}}(\nabla\bu^0\cdot\mathbf n^0)\cdot\bv\right]\\
&\\
&\displaystyle-\sum_{i=0}^4k^i\int_{\Gamma^i_{ext}}(\nabla\bu^i\cdot\mathbf n^i)\cdot\bv-\int_{\Gamma_{root}}(\nabla\bu^0\cdot\mathbf n^0)\cdot\bv
\end{array}$$


We now use the facts :
$$\begin{array}{ll}
\displaystyle\bigcup_{i=1}^4\Gamma^i_{int}=\Gamma^0_{int}&\\
\begin{array}{l}\mathbf n^i=-\mathbf n^0\\
-\nabla\bu^0\cdot\mathbf n^i=-k^i(\nabla\bu^i\cdot\mathbf n^i)\\
\end{array}&\text{ on }\Gamma^i_{int}, i=1,...,4\\
-k^i(\nabla\bu^i\cdot\mathbf n^i)=\text{Bi}~\bu^i&\text{ on }\Gamma_{ext}\\
\end{array}$$

yielding :

$$\begin{array}{rl}0=&\displaystyle\sum_{i=0}^4k^i\int_{\Omega^i}\nabla\bu^i\cdot\nabla\bv-[0]+\text{Bi}~\sum_{i=0}^4\int_{\Gamma^i_{ext}}\bu^i\cdot\bv-\int_{\Gamma_{root}}\bv
\end{array}$$





\end{document}